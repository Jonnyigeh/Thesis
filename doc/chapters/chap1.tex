\documentclass{subfiles}
\begin{document}
\chapter{Theory}
In this chapter we will cover the fundamental theory that we've built this thesis upon, starting from (what we believe to be) the most fundamental aspects in quantum mechanics.
There is an assumption that the reader have some basic knowledge of quantum physics, but this chapter will still follow a pedagogical approach, building from the ground up.
We start by introducing the fundamental Schrödinger equation, which naturally flows into the notion of a Hamiltonian operator and operators in general. From there, we move into 
second quantization, a useful transformation of the mathematical framework in quantum mechanics. In second quantization, we can easily build intuition and build understanding for
the following many-body quantum mechanical problems, and it's approximative solutions. Here we will cover some of the most applied approximations, and the most studied methods for solving
such systems, namely Hartree-Fock methods and Configuration Interaction. We will in this thesis work with atomic units, where $\hbar = m = e = 4\pi\epsilon_0 = 1$. For a detailed derivation of the conversion between atomic units and SI units, see appendix \ref{app:atomic_units}, and we will refer the reader to chapter 2.1.1 in \cite{szabo2012modern} for a more detailed introduction to atomic units.
\newpage
\subfile{../sections/chap1/lin_alg.tex}
\subfile{../sections/chap1/quantum_mechanics.tex}
\subfile{../sections/chap1/entanglement.tex}
\subfile{../sections/chap1/time_evolution.tex}
\subfile{../sections/chap1/quantum_computing.tex}
\end{document}