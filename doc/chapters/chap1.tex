\documentclass{subfiles}
\begin{document}
\chapter{Theory}\label{chap:2}
In this chapter, we lay out the theoretical foundations upon which this thesis is built, introducing what we consider to be the most essential aspects of quantum mechanics necessary for a clear understanding of the subsequent work. Although we presume that the reader possesses familiarity with quantum physics, we adopt a pedagogical approach throughout, gradually progressing from simple illustrative examples to more complex concepts. Our goal is to ensure the reader can comfortably follow the development of the project and grasp the methods and results presented in later chapters.

\begin{quote}
    \centering
    \textit{“Anyone who is not shocked by quantum theory has not understood it.”}\\
    \vspace{0.2cm}
    --- Niels Bohr
\end{quote}


We begin with a concise yet thorough review of the linear-algebraic framework underpinning quantum mechanics, including Dirac notation; operators and their properties; the concept of Hilbert spaces; and common basis sets. Collectively, these tools provide the language and mathematical structure necessary to describe quantum systems and their dynamics. Having established the framework, we then introduce the fundamental wave equation of quantum mechanics, the Schrödinger equation, which describes how quantum states evolve over time, including remarks on various numerical methods to solve this equation, like Hartree and Configuration Interaction theory. We discuss various numerical methods to solve this equation, notably Hartree and Configuration Interaction methods. Using these methods, we introduce the Morse double-well potential as our model quantum system and outline the theoretical framework underlying its (numerical) time evolution for analyzing the dynamics of an interacting two-particle system. Preliminary test results are provided to validate our numerical approaches and highlight the respective strengths of different methods in simulating quantum dynamics.

Next, we introduce the concept of entanglement, a uniquely quantum feature responsible for correlations between particles that defy classical explanations. We discuss how entanglement can be quantified and leveraged for quantum computational tasks, and we highlight the quantum mechanical phenomenon known as the avoided crossing.
Finally, we conclude the chapter by introducing quantum logic and control protocols essential for practical quantum computing implementations. We outline the basic principles of quantum gates and demonstrate how our quantum system can implement both single- and two-qubit gates. To evaluate these gates' effectiveness, we discuss the concept of gate fidelity, a measure of a quantum gate's accuracy in performing its intended operation. Additionally, we describe methods for implementing phase-corrected single-qubit rotations to achieve higher gate fidelities.

Throughout this thesis, we employ atomic units ($\hbar = m_e = e = 4\pi\epsilon_0=1.0$). For a detailed derivation relating atomic units to SI units, see Appendix \ref{app:appendix_A}; readers seeking further details may consult Chapter 2.1.1 in Szabo and Ostlund \cite{szabo1996modern}.


\newpage
\subfile{../sections/chap1/lin_alg.tex}
\subfile{../sections/chap1/quantum_mechanics.tex}
\subfile{../sections/chap1/time_evolution.tex}
\subfile{../sections/chap1/entanglement.tex}
\subfile{../sections/chap1/quantum_computing.tex}
\end{document}