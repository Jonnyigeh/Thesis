\documentclass{subfiles}
\begin{document}
\chapter{Concluding remarks and outlook}\label{chap:5}
In this thesis, we have explored the viability of a Morse-double well potential as a platform for quantum control of two-particle entanglement. We have viewed it as a candidate model for a prototype semi-conductor (or electronic device) quantum-dot qubit, aiming to demonstrate the feasibility of implementing one- and two-qubit quantum logic protocols. Our main goals were to (i) develop and validate a numerical framework for for obtaining the time-dependent Schrödinger equation (TDSE) solutions for the Morse double well in a product state approach, (ii) identify the optimal parameters for the potential to obtain optimized basis-set configurations for efficient representation, (iii) simulate coherent quantum-gate operations via time-evolution of the system, and (iv) assess the performance of the quantum control protocols in terms of gate fidelity, entanglement generation and population transfer.
\subsubsection*{Summary of key findings}
\begin{itemize}
    \item \textbf{Time-independent basis construction:} Our calculation demonstrated excellent agreement between the discrete-variable-representation (Sinc-DVR) basis and analytical expressions, with overlaps exceeding $99.99\%$ for the lowest six bound states \ref{fig:dvr_validation_overlap}, and energy deviations of less than $10^{-2}$ a.u. for the relevant energy levels \ref{fig:dvr_validation}.
    \item \textbf{Exchange interaction:} We showed that the exchange interaction has a negligible contribution to the ground-state energy for inter-well separations larger than $d = 10$ a.u. \ref{fig:exchange_shift}, confirming our assumption of strict locality and distinguishability of the particles. This justified the use of product states in our Hartree approach, which yielded a first excited-states energies within sub-meV of the exact configuration-interaction (CI) solution, and the remaining relevant energy levels with a deviance in the meV ranges of the exact solution ($10^{-4}$ a.u.).
    \item \textbf{Parameter optimization:} Running constrained optimization routines, we identified the optimal Morse double-well potential parameters for the two configurations $C_I$ and $C_{II}$. Configuration $C_I$ was optimized to yield distinct single-particle energies with wide energy gaps and zero entanglement, with the energy eigenstates showing complete overlap with the logical states $\ket{ij}$ as intended, indicated in Figure \ref{fig:populations_I}. Configuration $C_{II}$ was optimized to yield a degenerate first excited single-particle energy, allowing for coherent mixing of the logical states $\ket{10}$ and $\ket{01}$, with a high level of entanglement and a low energy gap between the first and second excited states. We aimed to achieve Bell-like entanglement in this configuration, and we see Bell-like mixing in the populations shown in Figure \ref{fig:populations_II}, with the populations of the logical states $\ket{10}$ and $\ket{01}$ being nearly equal, indicating a high degree of entanglement.
    \item \textbf{Time-dependent simulation:} With the optimal parameters in hand, we successfully simulated the time evolution of the system as we drove the system from configuration $C_I$ to $C_{II}$, and back again. We demonstrated coherent population transfer between the logical states $\ket{10}$ and $\ket{01}$. The time evolution was performed using the Crank-Nicolson method, indicating a successful implementation of the time-dependent Schrödinger equation in our numerical framework. 
    \item \textbf{Gate fidelity:} With the unitary two-qubit gate operation implemented, we calculated the gate fidelity for the transfer between the logical states $\ket{10}$ and $\ket{01}$, achieving classical fidelities exceeding $99.92\%$ for both gate operations after single-qubit phase correction was performed. For SWAP-gate showed an average fidelity of $98.79\%$, while the $\sqrt{\text{SWAP}}$ gate achieved an average fidelity of $77.30\%$. This verifies that while we achieved high fidelity population transfer, our lack of phase control during evolution resulted in a significant phase error, which is more prevalent in the $\sqrt{\text{SWAP}}$ gate due to complex phase factor in the gate.
\end{itemize}

\subsubsection*{Implications for Qubit architectures}
Our work positions the Morse double-well potential as an attractive model for quantum-dot qubits, demonstrating that it is possible to achieve coherent control of two-particle entanglement driven purely by Coulomb interaction via time-dependent Schrödinger evolution and perturbative control over the confining potential. By driving the gate operations purely by tuning the confining potential (well structure and separation), we achiveve high-fidelity gate operations without the need for additional external fields or complex control systems. This is a significant advantage, and can lead to simpler and more robust qubit architectures, reducing noise and decoherence that could arise from external control fields \cite{petta2005coherent, kuhlmann2013charge, yoneda2018quantum}.

The ability to optimize the potential parameters for distinct configurations allows for flexible control over the system, enabling the implementation of quantum logic gates and entanglement generation protocols. The successful demonstration of coherent mixing of logical states in configuration $C_{II}$ indicates that the Morse double-well potential can serve as a viable platform for quantum information processing, with potential applications in quantum computing and quantum communication \cite{nielsen2010quantum}. The Morse potentials mathematical form, originally derived to model the anharmonic bonds in diatomic molecules\cite{morse1929diatomic}, naturally describes a smooth, tunable well profile that can be engineered in semiconductor heterostructures. Adjusting gate voltages one can closely mimic the desired potential (like in \cite{leinonen2024coulomb}), allowing for precise control over well depths, barrier heights, and inter-well separations. This tunability not only leverages decades of experience in molecular-potential shaping, but also aligns with existing semiconductor fabrication techniques that routinely create similiar confinement profiles in quantum dots \cite{jacak2013quantum, garcia2021semiconductor}.

Because the Morse double-well, and the entirety of the gate operation, can be implemented purely by electrostatic control, it offers a direct path from theory to experiment: One can directly translate the optimized parameters into gate voltages in a quantum dot system, without the need for complex microwave or laser control systems. Consequently, the work presented in this thesis provides a concrete foundation for future efforts to build and characterize Morse-based quantum-dot qubits, and to explore their potential for scalable quantum computing architectures. 

In short, the deep connection between the Morse potential and real-world fabrication, combined with the promising fidelities and reconfigurability we have demonstrated, make this double-well potential a compelling model for future scalable, high-performance quantum-dot qubit architectures. 

\subsubsection*{Limitations of our study}
\begin{itemize}
    \item \textbf{Dimensionality and spin degrees of freedom:} Our 1D model neglects transverse confinement and spin-orbit coupling, both of which can significantly alter exchange interaction and coherence properties in real devices \cite{kuhlmann2013charge, yoneda2018quantum}. Furthermore, the 1D model does not capture the full complexity of a real 3D quantum dot system, and future work should consider extending the model to include more degrees of freedom.
    \item \textbf{Environment and noise effects:} We did not incorporate environmental effects such as phonon interactions, charge-noise models, or other decoherence mechanism that are known to limit coherence in real quantum-dot systems \cite{jacak2013quantum, nielsen2010quantum}. These effects can significantly impact the performance of quantum control protocols.
    \item \textbf{Phase control:} While achieving high fidelity population transfer, we did not implement a sophisticated phase control protocol during the time evolution. This resulted in significant phase errors, particularly in the $\sqrt{\text{SWAP}}$ gate operation, which we were unsuccessful in correcting with single-qubit phase corrections. 
\end{itemize}

\subsubsection*{Outlook and recommendations for future work}
Future work should focus on extending the Morse double-well model to include more realistic features such as transverse confinement, spin-orbit coupling, and environmental effects. It should also extend the model to all three dimensions to capture the full complexity of a real quantum dot system. Additionally, it would be beneficial to explore the implementation of more advanced quantum control protocols, such as multi-qubit gates and error correction schemes beyond single-qubit rotations post-simulation, to further enhance the performance of the Morse double-well potential as a quantum-dot qubit architecture. Our work barely scraped the surface, and there are many avenues for further research into multi-qubit entanglement generation and quantum control protocols in Morse-like potentials. 

Building on our 1D model, one should consider more refined methods for both representing the quantum state, and methods for time evolution. For example, our limited system truncated to a moderate number of basis states, and a further Hartree reduction to build product states. This could be improved upon by including a larger basis set and using more sophisticated methods for solving the quantum system without the need for such aggressive truncation. This could include using more advanced numerical methods such as the time-dependent density functional theory (TDDFT) that would allow for more complex interactions and higher order correlations between particles \cite{car1985unified, hafner2008ab}. A continued study in our simplified 1D model would want to explore more direct methods to control the phase evolution of the system, such as tracking relative phases of the logical states through the evolution, and using this to achieve higher fidelity gate operations. This could be done by implementing a more sophisticated ramping protocol that takes into account the relative phases of the logical states, and optimizes the ramping speed and duration to achieve the desired gate operation.

Finally, it would be interesting to explore the physical implementation of the Morse double-well potential in real quantum dot systems, and to investigate the feasibility of implementing the quantum control protocols demonstrated in this thesis in a real experimental setup. This would require a detailed study of the physical parameters of the Morse potential, and how they can be tuned in a real quantum dot system \cite{garcia2021semiconductor}. It would also require careful consideration in choice of materials and fabrication techniques to ensure that the Morse double-well potential can be realised in practice. Following the discussions in Appendix \ref{app:appendix_A}, we believe that the Morse potential can be implemented in a semi-conductor quantum dot system, and that our simulation parameters are within realistic ranges for quantum dot systems.

\subsubsection*{Concluding remarks}
In conclusion, we have demonstrated—with phase corrected SWAP fidelities exceeding $98\%$—that a Morse double-well potential offers a fully electrostatic platform for high-fidelity, two-particle entanglement protocols. By meeting our four key objectives (numerical framework validation, parameter optimizationi, coherent gate simulation, and fidelity assessment), this works lay a solid foundation for future research into experimentally accessible quantum-dot qubits based on the Morse double-well. Looking ahead, translating these theoretical insights into practical device architectures and incorporating spin degrees of freedom will be critical next steps toward realizing robust, scalable Morse-based quantum-computing systems.
\end{document}