\documentclass[subfiles]
\begin{document}
\chapter{Concluding remarks and outlook}
In this thesis, we have explored the viability of a Morse-double well potential as a platform for quantum control of two-particle entanglement. We have viewed it as a candidate for a prototype semi-conductor quantum-dot qubit, aiming to demonstrate the feasibility of implementing one- and two-qubit quantum logic protocols. Our main goals were to (i) develop and validate a numerical framework for for obtaining the time-dependent Schrödinger equation (TDSE) solutions for the Morse double well in a product state approach, (ii) identify the optimal parameters for the potential to obtain optimized basis-set configurations for efficient representation, (iii) simulate coherent quantum-gate operations via time-evolution of the system, and (iv) assess the performance of the quantum control protocols in terms of gate fidelity, entanglement generation and population transfer.
\subsubsection*{Summary of key findings}
\begin{itemize}
    \item \textbf{Time-independent basis construction:} Our calculation demonstrated excellent agreement between the discrete-variable-representation (Sinc-DVR) basis and finite-difference representation (FDR) exact energy eigenbasis, with overlaps exceeding $99.99\%$ for the lowest six bound states \ref{fig:dvr_validation_overlap}, and energy deviations of less than $10^{-2}$ a.u. \ref{fig:dvr_validation}\textcolor{red}{TODO: Convert to effective msas units, 1 a.u. = 11 meV}
    \item \textbf{Exchange interaction:} We showed that the exchange interaction has a negligible contribution to the ground-state energy for inter-well separations larger than $d = 10$ a.u. \ref{fig:exchange_shift}, confirming our assumption of strict locality and distinguishability of the particles. This justified the use of product states in our Hartree approach, which yielded a ground-state energy within sub-meV of the exact configuration-interaction (CI) solution.\textcolor{red}{TODO: Check the energy deviation is correct.}
    \item \textbf{Parameter optimization:} Running constrained optimization routines, we identified the optimal Morse double-well potential parameters for the two configurations $C_I$ and $C_{II}$. Configuration $C_I$ was optimized to yield distinct single-particle energies with wide energy gaps and zero entanglement, with the energy eigenstates showing complete overlap with the logical states $\ket{ij}$ as intended, indicated in Figure \ref{fig:state_populations_I}. Configuration $C_{II}$ was optimized to yield a degenerate first excited single-particle energy, allowing for coherent mixing of the logical states $\ket{10}$ and $\ket{01}$, with a high level of entanglement and a low energy gap between the first and second excited states. We aimted to achieve Bell-like entangled in this configuration, and we see Bell-like mixing in the populations shown in Figure \ref{fig:state_populations_II}, with the populations of the logical states $\ket{10}$ and $\ket{01}$ being nearly equal, indicating a high degree of entanglement.
    \item \textbf{Time-dependent simulation:} 
    \item \textbf{Gate fidelity:}
\end{itemize}

\subsubsection*{Implications for Qubit architectures}
Our results positions the Morse double-well potential as an attractive model for quantum-dot qubits, demonstrating that it is possible to achieve coherent control of two-particle entanglement driven purely by Coulomb interaction via time-dependent Schrödinger evolution and perturbative control over the one-body potential. The ability to optimize the potential parameters for distinct configurations allows for flexible control over the system, enabling the implementation of quantum logic gates and entanglement generation protocols. The successful demonstration of coherent mixing of logical states in configuration $C_{II}$ indicates that the Morse double-well potential can serve as a viable platform for quantum information processing, with potential applications in quantum computing and quantum communication. The similiarity of the Morse potential to molecular potentials, and the fact that it can be implemented in semi-conductor quantum dots, makes it a promising candidate for future quantum-dot qubit architectures. The results of this thesis provide a solid foundation for further research into the implementation of quantum control protocols in Morse double-well potentials, and the development of more advanced quantum-dot qubit architectures. \textcolor{red}{TODO: Review, and extend this section.}

\subsubsection*{Limitations of our study}
\begin{itemize}
    \item \textbf{Dimensionality and Spin degrees of freedom:} Our 1D model neglects transverse confinement and spin-orbit coupling, both of which can significantly alter exchange interaction and coherence properties in real devices. Furthermore, the 1D model does not capture the full complexity of a real 3D quantum dot system, and future work should consider extending the model to include more degrees of freedom.
    \item \textbf{Environment and Noise effects:} We did not incorporate environmental effects such as phonon interactions, charge-noise models, or other decoherence mechanism that are known to limit coherence in real quantum-dot systems\cite{someone}. These effects can significantly impact the performance of quantum control protocols.
    \item \textbf{Phase control:}
\end{itemize}

\subsubsection*{Outlook and recommendations for future work}
Future work should focus on extending the Morse double-well model to include more realistic features such as transverse confinement, spin-orbit coupling, and environmental effects. It should also extend the model to all three dimensions to capture the full complexity of a real quantum dot system. Additionally, it would be beneficial to explore the implementation of more advanced quantum control protocols, such as multi-qubit gates and error correction schemes, to further enhance the performance of the Morse double-well potential as a quantum-dot qubit platform. Our work barely scraped the surface, and there are many avenues for further research into multi-qubit entanglement generation and quantum control protocols in Morse potentials. 

Building on our 1D model, one should consider more refined methods for both representing the quantum state, and methods for time evolution. For example, our limited system truncated to a moderate number of basis states, and a further Hartree reduction to build product states. This could be improved upon by including a larger basis set and using more sophisticated methods for solving the quantum system without the need for such aggressive truncation. This could include using more advanced numerical methods such as the time-dependent density functional theory (TDDFT) that would allow for more complex interactions and higher order correlations between particles. A continued study in our simplified 1D model would want to explore more direct methods to control the phase evolution of the system, such as tracking relative phases of the logical states through the evolution, and using this to achieve higher fidelity gate operations. This could be done by implementing a more sophisticated ramping protocol that takes into account the relative phases of the logical states, and optimizes the ramping speed and duration to achieve the desired gate operation.

Finally, it would be interesting to explore the physical implementation of the Morse double-well potential in real quantum dot systems, and to investigate the feasibility of implementing the quantum control protocols demonstrated in this thesis in a real experimental setup. This would require a detailed study of the physical parameters of the Morse potential, and how they can be tuned in a real quantum dot system. It would also require careful consideration in choice of materials and fabrication techniques to ensure that the Morse double-well potential can be realised in practice.

\subsubsection*{Concluding remarks}
In conclusion, this thesis has demonstrated the feasibility of using a Morse double-well potential as a platform for quantum control of two-particle entanglement. We have shown that it is possible to achieve coherent control of two-particle entanglement driven purely by Coulomb interaction via time-dependent Schrödinger evolution and perturbative control over the one-body potential. Disregarding the limitations of our study and the simplifications made, we believe that our results provide a solid foundation for further research into the implementation of quantum control protocols in Morse-like potentials, and the development of more advanced quantum-dot qubit architectures. 

\end{document}