\documentclass{subfiles}
\begin{document}
\chapter{Introduction}
\section{Background and motivation}
Quantum computers promise to revolutionize modern computations by leveraging quantum mechanical phenomena such as superposition and entanglement to solve problems that are intractable for classical computers. By harnessing these phenomena, it has been demonstrated that quantum computers can outperform their classical counterparts in specific tasks, such as factoring large numbers, simulating quantum systems and cryptography \cite{shor1999polynomial, shor1996fault, gisin2002quantum, grover1996fast}. The rapid development of quantum algorithms and error-correction techniques has fueled optimism that we are nearing the threshold for practical 'quantum advantage'—where quantum computers outperform classical systems in real-world applications—but significant challenges remain in the construction of scalable, noise-resilient quantum hardware  \cite{daley2022practical, lau2022nisq,}.

Quantum-dot platforms have emerged as a leading candidate for scalable quantum computing due to their compatibility with semiconductor fabrication and the potential for on-chip qubit integration \cite{burkard1999coupled, loss1998quantum}. These semiconductor materials offer electronic and optical properties that can be precisely engineered via their composition and crystal-lattice structure, to realize any desired quantum-dot architecture. This has allowed for the development of technologies from mobile phones to satellites, and they now seem poised to reshape modern quantum computing hardware \cite{garcia2021semiconductor, zhang2018qubits}. Due to the nanometer scale of semiconductor quantum dots, the electronic motion leads to atom-like electronic structures, where the discrete energy levels can be tuned by varying the quantum dot size, shape, and surrounding environment \cite{terna2021future}. Quantum dot spin-qubits have previously demonstrated long coherence times, but their manipulation via electric fields is too slow for many future applications \cite{stano2022review}, and they are susceptible to noise from spin-orbit and hyperfine interactions \cite{kuhlmann2013charge, yoneda2018quantum}. To overcome these limitations, researchers are looking into alternative approaches that utilize electrostatic control \cite{veldhorst2014addressable, weber2012engineering}, and some are looking at alternative quantum dot qubits based on electric charge states \cite{gorman2005charge, kim2015microwave}. Much progress has been made in the development of quantum-dot qubits, and the field is rapidly evolving, with new techniques showing high fidelity gate operations and long coherence times \cite{garcia2021semiconductor}.

Yet, implementing high-fidelity gate operations in such systems often demands complex control hardware—microwave drives, rapid voltage pulses, and magnetic field gradients—making them susceptible to charge noise, phase instabilities, and crosstalk \cite{xue2022quantum, kuhlmann2013charge, heinz2021crosstalk}. Purely electrostatic approaches, which manipulate qubit states by shaping static or slowly varying potential landscapes, promise simpler, lower-noise architectures \cite{veldhorst2014addressable, leinonen2024coulomb}. The Morse double-well potential, with its tunable anharmonic profile and direct analogy to molecular bonding, offers a natural testbed for these electrostatic quantum control strategies—due to the already atom-like electronic structures commonly found in quantum dot systems. Originally developed to model molecular interactions \cite{morse1929diatomic}, and widely used in quantum chemistry to model diatomic molecules \cite{piela2006ideas}, we believe the Morse potential to be a promising candidate for quantum-dot qubit architecture. Some studies have explored the Morse potential in the context of quantum computing of molecules \cite{apanavicius2021morse}, but it is generally underexplored for quantum-dot qubits and its potential role in quantum hardware. Here we present the Morse double-well potential as a candidate quantum-dot qubit architecture, and show that it can be engineered to create a tune-able two-qubit system, where the anharmonicity of the potential allows for efficient population transfer and entanglement generation between two qubits. Moreover, because the anharmonic energy spacings can be independently tuned (so no two transitions are degenerate), this design also supports stable single-qubit operations.

In this thesis, we develop and analyze a one-dimensional model of two interacting electrons in a Morse double-well potential—a form that can be realized either by static split-gate voltages in a semiconductor heterostructure or by equivalent electrostatic barrier shapes in an experimental electronics setup \cite{smet2002gate, leinonen2024coulomb}. By tuning those voltages to set the well's separation, depth, and width, we engineer a potential landscape that supports two well-localized states and switch between an idle mode (large single-particle gaps, suppressed entanglement) and a gate mode (near-degenerate first excited levels inducing two-particle entanglement). Using this platform, we demonstrate coherent SWAP and $\sqrt{\text{SWAP}}$ protocols driven purely by electrostatic tuning of the Morse well, achieving high-fidelity population transfer and entanglement generation once appropriate single-qubit phase corrections are applied.

\section{Goals}
Achieving robust, electrostatic gates in a Morse double-well potential requires both methodological rigour and quantitative benchmarks. First, we must validate a numerical solver that captures both static and dynamic behavior of two-electron systems to high precision. Next, we need to identify optimal parameter configurations that isolate logical states, yet allow for fast entangling interaction. Having defined those, we must implement concrete gate protocols in time-domain simulations and finally, we must measure gate performance through standardized fidelity metrics to assess our implementation.

The thesis is structured around four main objectives:
\begin{enumerate}
\item \textbf{Numerical framework development}: Combine a grid based Sinc-Discrete Variable Representation (Sinc-DVR) with a product-state Hartree approximation to accurately solve both the time-independent and time-dependent Schrödinger equations for two interacting particles confined in a Morse double-well potential.
\item \textbf{Parameter optimization}: Identify two operational configurations—an \textit{idle} mode with large single-particle energy gaps and suppressed entanglement, and a \textit{gate} mode that enforces near-degeneracy of the first excited states to induce two-particle entanglement and population transfer.
\item \textbf{Time-evolution simulations}: Implement and test coherent SWAP and $\sqrt{\text{SWAP}}$ gate protocols by dynamically tuning the one-body potential, monitoring population transfer, entanglement generation and stability.
\item \textbf{Performance assessment}: Quantify gate performance through classical and average (coherent) fidelity measures, and apply appropriate phase corrections, and assess state leakage outside the logical subspace.
\end{enumerate}

\section{Thesis structure}
The remainder of the thesis is organized as follows. In Chapter \ref{chap:2}, we review the theoretical foundations necessary for our work: such as Dirac notation, operator formalism, and basis-set expansions; the Hartree and configuration-interaction methods; the Morse potential model; time-evolution operator theory; measures of entanglement; and fidelity metrics. Chapter \ref{chap:3} describes our computational methods and implementation details, including system initialization, Sinc-DVR basis validation, the Hartree product approximation with configuration-interaction corrections, numerical propagators (Crank-Nicolson, matrix exponentiation, and ramp protocols), and the post-processing steps for phase correction and fidelity extraction. In Chapter \ref{chap:4}, we present our results, starting with intermediate benchmarks of exchange shifts, DVR accuracy, and propagator performance; followed by optimized Morse-double-well configurations for the idle and gate modes; and concluding with time-domain simulations of SWAP and $\sqrt{\mathrm{SWAP}}$ protocols along with detailed fidelity and entanglement analyses. Finally, Chapter \ref{chap:5} offers concluding remarks and outlook, summarizing the key findings, discussing implications for quantum-dot qubit architectures, outlining study limitations, and proposing directions for future work. Supplementary derivations and discussions—such as effective-mass conversions and the midpoint Hamiltonian formulation for Crank-Nicolson propagation—are collected in Appendix \ref{app:appendix_A}.
\end{document}