\documentclass{subfiles}
\begin{document}
\begin{abstract}
    Phase instability, control-signal complexity, and fabrication remain key obstacles to implementing high-fidelity, scalable quantum-dot qubits. In this work, we explore the Morse double-well potential as a model for two-electron quantum-dot qubit and develop a numerical framework to solve its time-dependent Schrödinger equation. The method is validated by $>99.99\%$ agreement between the Sinc-Discrete Variable Representation (Sinc-DVR) basis used and the exact eigenstates of the Morse potential. We employ a product state approach to the two-electron system, allowing us to treat the two electrons as independent particles confined in a double-well potential, with vanishing exchange interaction at our chosen inter-well separation of $d=15$ a.u. Using our framework, we identify optimal double-well parameters for two distinct operating configurations: a measurement mode with well-separated single-particle levels (zero entanglement) and a gate mode with nearly degenerate first excited single-particle states (maximal entanglement), ensuring that the coupling term in the system Hamiltonian is significant (or non-negligible). Time-domain simulations of the two-electron system, driving the system adiabatically between configurations, demonstrate high-fidelity two-particle entanglement via purely electrostatic control of the potential: detailed benchmarks show a SWAP operation attaining average fidelity of $98.79\%$ and classical fidelity of $99.98\%$ after single-qubit phase corrections are applied. These results establish the Morse double-well as a viable quantum-dot qubit platform, and lay the groundwork for future extensions incorporating spin degrees of freedom and realistic noise models.

\end{abstract}
\newpage
\section*{Acknowledgements}
I would like to express my heartfelt gratitude to my co-supervisor Oskar Leinonen for his invaluable guidance, support, and encouragement throughout this project. I would also like to thank my supervisor Morten Hjorth-Jensen for his insightful feedback, for providing me with the opportunity to work on this project, and for taking me under his wing at the quantum computing section. Without you two, there would be no thesis, and for that I am truly grateful. I would also like to extend my appreciation to all my friends, co-students and colleagues at CCSE for making my time at the office so enjoyable and inspiring. To all the joyful lunch breaks, table tennis matches, and quizzes, thank you. The struggles of writing this thesis was enjoyable in your company. Last, but not least, my deepest gratitude goes to my mother for pushing me, not letting me give up and for always telling me to achieve the things I am capable of. I would not sit here today without your support and encouragement. Thank you.

\end{document}
