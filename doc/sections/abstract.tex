\documentclass{subfiles}
\begin{document}
\begin{abstract}
    Phase instability, control-signal complexity, and fabrication remain key obstacles to implementing high-fidelity, scalable quantum-dot qubits. In this work, we explore the Morse double-well potential as a model for two-electron quantum-dot qubit and develop a numerical framework to solve its time-dependent Schrödinger equation. The method is validated by $>99.99\%$ agreement between the Sinc-Discrete Variable Representation (Sinc-DVR) basis used and the exact eigenstates of the Morse potential. We employ a product state approach to the two-electron system, allowing us to treat the two electrons as independent particles in a double-well potential, with vanishing exchange interaction at our chosen inter-well separation of $d=15$ a.u. Using our framework, we identify optimal double-well parameters for two distinct operating configurations: a measurement mode with well-separated single-particle level (zero entanglement) and a gate mode with nearly degenerate first excited single-particle states (maximal entanglement) to enable coupling between the two particles. Time-domain simulations of the two-electron system, driving the system adiabatically between configurations, demonstrate high-fidelity two-particle entanglement via purely electrostatic control of the potential: detailed benchmarks show a SWAP operation attaining average fidelity of $98.79\%$ and classical fidelity of $99.98\%$ after single-qubit phase correction. These results establish the Morse double-well as a viable quantum-dot qubit platform, and lay the groundwork for future extensions incorporating spin degrees of freedom and realistic noise models.

\end{abstract}
\end{document}
