\documentclass{subfiles}
\begin{document}
\section{Quantum Logic and Control}
Quantum computing is a rapidly developing field that promises to revolutionize information processing across disciplines such as cryptography, optimization, and materials science. While the subject has recently gained popular attention — even being mentioned in political discourse such as the 2024 U.S. presidential debates — the practical realization of quantum computers still faces several significant challenges \cite{lau2022nisq}. At the top among these are the development of robust quantum hardware and the precise control of quantum systems, both of which fall under the umbrella of \emph{quantum control}\cite{huang1983controllability, d2021introduction}. \\ 

In this work, we study a simple double-well Morse potential — a possible prototype for a quantum dot system — and investigate its time-evolution under the influence of Coulomb interaction. By tuning the parameters that define the potential, we perturb the system and observe how its dynamics evolve. This controlled perturbation forms the basis for implementing our quantum control protocols.
% QUANTUM CONTROL
\subsection{Quantum Control}
Quantum control refers to the ability to manipulate the state of a quantum system in a precise and predictable way. It is a foundational requirement for quantum computing, where qubits (quantum bits) must be steered through specific trajectories in their Hilbert space to perform logic operations. Just as classical computation depends on switching transistors (bits) on and off, quantum computation depends on the coherent manipulation of qubits via unitary transformations \cite{nielsen2010quantum}. \\ 
In this context, our double well Morse potential serves as a minimal platform to demonstrate basic quantum control principles. By applying external fields or, as we shall see in this work, by tuning the parameters of the potential we can prepare our system for single-qubit operations, and induce transitions between different quantum states to perform two-qubit operations. This paves the way for considering Morse-type potentials candidates for robust and tunable qubit architectures - an avenue that, to our knowledge, has yet to be extensively explored in the literature.\textcolor{red}{TODO: This is not exactly true - we should search more and instead relate our study to existing litterature.} \\ 

Achieving reliable single-qubit operations in our system requires energy levels to be well-separated and non-degenerate. This is crucial so that one can apply external fields to induce transitions in a single qubit without inducing unwanted transitions in the other qubit. With the 4 tunable parameters in our double well Morse potential \eqref{eq:double_well_morse_potential}, we can achieve this by finding such a configuration that dinstinguishes the lowest lying single particle energy levels. The single particle energies are the Hartree energies of the two subsystems, found through the Hartree calculation\ref{sec:bipartite_hartree}. In our work, this shall be referenced as the \emph{measurement configuration}, or \emph{configuration I}. It is in this configuration we want to make measurements of the system. 
\\ 

The second configuration, \emph{configuration II}, aims to induce a transition between the two lowest lying single particle states. This configuration make the first excited states of both subsystems degenerate, while maintaining separation among higher levels. In this configuration, we expect the respective energy eigenstates to mix, and this will entangle our two qubits - which in turn, allows for two-qubit operations such as the Swap gate, or the CNOT gate to mention a few\cite{leinonen2024coulomb, nichol2017high}.
%% QUANTUM LOGIC
\subsection{Quantum Logic}
\subsubsection*{Two-qubit operations}
\textcolor{red}{TODO: Re-write so that we focus only on SWAP gate unless we get dcent results rotating to the iSWAP gate.}
In our model, we aim to emulate the Swap- and $\sqrt{\text{SWAP}}$-gate by dynamically tuning the system into configuration II as previously described. The Swap gate is a two-qubit gate that swaps the states of two qubits \cite{nielsen2010quantum}. It is defined as:
\begin{equation}
    \text{Swap} = \begin{pmatrix}
    1 & 0 & 0 & 0 \\
    0 & 0 & 1 & 0 \\
    0 & 1 & 0 & 0 \\
    0 & 0 & 0 & 1
    \end{pmatrix} \label{eq:swap_gate},
\end{equation}
and we have the square root of the Swap gate, $\sqrt{\text{Swap}}$, which maximally entangles the affected qubits, defined as:
\begin{equation}
    \sqrt{\text{Swap}} = \begin{pmatrix}
    1 & 0 & 0 & 0 \\
    0 & \frac{1+i}{2} & \frac{1-i}{2} & 0 \\
    0 & \frac{1-i}{2} & \frac{1+i}{2} & 0 \\
    0 & 0 & 0 & 1
    \end{pmatrix} \label{eq:sqrt_swap_gate}.
\end{equation}
The ordinary Swap gate simply exchanges the states of two qubits without introducing any correlations, whereas its square-root variant $\sqrt{\text{Swap}}$, split amplitude coherently between the two qubits, therefore turning a separable state input, $\ket{01}$ or $\ket{10}$, into a maximally entangled Bell state \eqref{eq:bell_states}. These states are indispensable resources for many quantum algorithms and protocols, such as quantum teleportation, entanglement-based cryptography and many error-correcting codes \cite{nielsen2010quantum, bouwmeester1997experimental, yin2020entanglement}. 
\\
In our work, configuration $C_{II}$ is designed to bring the first excited states of the two subsystmems into resonance, which allows for a coherent exchange of population between the qubit states. This is due to the Coulomb interaction between the particles at this near-degenerate condition, akin to the avoided crossing shown in Section \ref{sec:avoided_crossings}. Although we will not explicitly model the phase evolution needed to realize the exact Swap or $\sqrt{\text{SWAP}}$ matrix, the behavior of our system mimics the essential ingredients of a Swap-like gate: resonant exchange and entanglement generation. \\
By controlling the duration for which the system remains in the various configurations, and the time spent ramping between them, we can identify time windows where the system approximates the desired state-swapping behavior. This is analogous to how SWAP-gates are implemented in superconducting or trapped-ion systems through Hamiltonian engineering and pulse timing \cite{picard2025entanglement}. \\

Thus, our model represents a novel first step towards demonstrating quantum logic operations using Morse-type double well potentials. The anharmonicity of the Morse potential enables isolation of low-lying states and avoids (unwanted) higher-level leakage, which is essential for the construction of robust, tunable qubit architectures\cite{nielsen2010quantum}.

\subsubsection*{Single-qubit operations}\ref{sec:single_qubit_operations}
In addition to our two-qubit operations, high-fidelity quantum logic also requires precise single-qubit operations.  In experimental practice, single-qubit gates are implemented by applying tailored external fields or voltage pulses that transiently shift one dot's energy levels \cite{mcdermott2014accurate}. In our Morse double-well model, configuation $C_I$ is engineered to have well-separated single-particle energies in the two subsystems. By driving at the resonance of one qubit, we can induce rotations in the state of that qubit without affecting the other. 

In our study we focus mainly on \emph{Z-rotations}, which are rotations around the $Z$-axis of the Bloch sphere, which crucially can correct unwanted dynamical phases that may accumulate during a two-qubit operation. A Z-rotation is defined as:
\begin{equation}
    R_Z(\theta) = \begin{pmatrix}
    e^{-i\theta/2} & 0 \\
    0 & e^{i\theta/2}
    \end{pmatrix} \label{eq:z_rotation_gate},
\end{equation}
where $\theta$ is the rotation angle. Written more compactly using the Pauli $Z$ operator, this can be expressed as $R_z(\theta) = \text{exp}(-i\theta Z/2)$ \cite{nielsen2010quantum}. The ultimate goal of these single-qubit operations is to increase the fidelity of our two-qubit operations, such as the SWAP gate, by correcting for any unwanted relative phases that misaligns the evolved statse with the target states. We note that $X$ and $Y$ rotations similarly can be performed, but our present work is limited to Z-rotations for phase correction.

\subsection{Fidelity measures}\ref{sec:fidelity_measures}
To quantify, and assess the performance of our quantum control protocols we will measure the \emph{fidelity} of the numerically generated two-qubit operation $U = \Psi_0^\dagger\Psi(t_f)$ against the ideal target operation $U_{target}$, where the target operation are the two-qubit gate matrices e.g \eqref{eq:swap_gate}. 
We introduce the following fidelity measures:
\begin{itemize}
    \item \textbf{Average fidelity:} This is the standard measure of fidelity in quantum gate operations, sensitive to both population errors and incorrect phases. Following the derivations in Pedersen et al. \cite{pedersen2007fidelity}, in a $d-$dimensional Hilbert space the average fidelity is defined as:
    \begin{equation*}
    F_{avg}(U, U_{target}) = \frac{1}{d(d+1)}\bigg[\text{Tr}(MM^\dagger) + |\text{Tr}(M)|^2\bigg],
    \end{equation*}
    where $M = U_{target}^\dagger U$. Assuming that $U$ is unitary, $M$ is therefore unitary, and $\text{Tr}(MM^\dagger) = d$, meaning the average fidelity is given by:
    \begin{equation}
        F_{avg}(U, U_{target}) = \frac{1}{d(d+1)}\bigg[d + |\text{Tr}(U_{target}^\dagger U)|^2\bigg]\label{eq:avg_fidelity}.
    \end{equation}
    \item \textbf{Classical fidelity:} If we are only interested in the population transfer, and not the relative phases, then we may compare the two classical transition matrices $P$ and $P_{target}$, whose entries are $P_{ij} = |U_{ij}|^2$, through the Bhattacharyya overlap \cite{bhattacharyya1943measure}:
    \begin{equation*}
        F_j(P, P_{target}) = \sum_{i,j} \sqrt{P_{ij} P_{target,ij}},
    \end{equation*}
    and then average over all inputs $j$,
    \begin{equation}
        F_{classical}(U, U_{target}) = \frac{1}{d}\sum_j F_j(P, P_{target})\label{eq:classical_fidelity}. 
    \end{equation} 
    Classical fidelity measures how well the gate operation reproduces the target population transfer, disregarding any phase information \cite{nielsen2010quantum}. For deterministic gates - where each input state $\ket{j}$ is mapped to a unique output $\ket{g(j)}$ - the target transition matrix probabilities satisfy $P_{target,ij} = \delta_{g(j),i}$, where $\delta_{ij}$ is the Kronecker delta. The Bhattacharyya overlap then simplifies to
    \begin{equation*}
        F_j(P, P_{target}) = \sum_i \sqrt{P_{ij} \delta_{g(j),i}} = |U_{g(j), j}|.
    \end{equation*}
    and the classical fidelity reduces to the average of the absolute values of the transition amplitudes:
    \begin{equation}
        F_{classical}(U, U_{target}) = \frac{1}{d}\sum_j |U_{g(j), j}|^2 = \frac{1}{d}\sum_j P_{g(j), j}\label{eq:classical_fidelity_deterministic}.
    \end{equation}
    This is the average probability of correctly transferring the population from input state $\ket{j}$ to the target state $\ket{g(j)}$. 
\end{itemize}
To summarize, the two fidelity measures introduced above play complementary roles in assessing the performance of our quantum control protocols. The average fidelity (Equation \eqref{eq:avg_fidelity}) captures the full coherent performance of theg gate, including both population transfer and phase alignment, while the classical fidelity (Equation \eqref{eq:classical_fidelity}) isolates purely population transfer, disregarding any phase information. Initially, we will primarily focus on the classical fidelity to asseess if our two-qubit operations achieve the desired population transfer. Once the population transfer is satisfactory, we will calculate the average fidelity to assess the overall performance of our two-qubit operations, including any phase errors. This will then be usde to find suitable single-qubit operations that can correct for any unwanted phases that may have accumulated during the two-qubit operation, resulting in a better average fidelity measure, and a more successful two-qubit operation. In the method section, we will outline how we compute these fidelity measures numerically from the time-evolved system, and how this guides the construction of our single-qubit $R_Z$ corrections to yield a high-fidelity two-qubit gate.


\end{document}