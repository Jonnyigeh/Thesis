\documentclass{subfiles}
\begin{document}
\section{Quantum Logic and Control}
Quantum computing is a rapidly developing field that promises to revolutionize information processing across disciplines such as cryptography, optimization, and materials science. While the subject has recently gained popular attention — even being mentioned in political discourse such as the 2024 U.S. presidential debates — the practical realization of quantum computers still faces several significant challenges \cite{lau2022nisq}. At the top among these are the development of robust quantum hardware and the precise control of quantum systems, both of which fall under the umbrella of \emph{quantum control}\cite{huang1983controllability, d2021introduction}. \\ 

In this work, we study a simple double-well Morse potential — a possible prototype for a quantum dot system — and investigate its time-evolution under the influence of Coulomb interaction. By tuning the parameters that define the potential, we perturb the system and observe how its dynamics evolve. This controlled perturbation forms the basis for implementing our quantum control protocols.
% QUANTUM CONTROL
\subsection{Quantum Control}
Quantum control refers to the ability to manipulate the state of a quantum system in a precise and predictable way. It is a foundational requirement for quantum computing, where qubits (quantum bits) must be steered through specific trajectories in their Hilbert space to perform logic operations. Just as classical computation depends on switching transistors (bits) on and off, quantum computation depends on the coherent manipulation of qubits via unitary transformations \cite{nielsen2010quantum}. \\ 
In this context, our double well Morse potential serves as a minimal platform to demonstrate basic quantum control principles. By applying external fields or, as we shall see in this work, by tuning the parameters of the potential we can prepare our system for single-qubit operations, and induce transitions between different quantum states to perform two-qubit operations. This paves the way for considering Morse-type potentials candidates for robust and tunable qubit architectures - an avenue that, to our knowledge, has yet to be extensively explored in the literature.\textcolor{red}{TODO: This is not exactly true - we should search more and instead relate our study to existing litterature.} \\ 

Achieving reliable single-qubit operations in our system requires energy levels to be well-separated and non-degenerate. This is crucial so that one can apply external fields to induce transitions in a single qubit without inducing unwanted transitions in the other qubit. With the 4 tunable parameters in our double well Morse potential \eqref{eq:double_well_morse_potential}, we can achieve this by finding such a configuration that dinstinguishes the lowest lying single particle energy levels. The single particle energies are the Hartree energies of the two subsystems, found through the Hartree calculation\ref{sec:bipartite_hartree}. In our work, this shall be referenced as the \emph{measurement configuration}, or \emph{configuration I}. It is in this configuration we want to make measurements of the system. 
\\ 

The second configuration, \emph{configuration II}, aims to induce a transition between the two lowest lying single particle states. This configuration make the first excited states of both subsystems degenerate, while maintaining separation among higher levels. In this configuration, we expect the respective energy eigenstates to mix, and this will entangle our two qubits - which in turn, allows for two-qubit operations such as the iSwap gate, or the CNOT gate to mention a few\cite{leinonen2024coulomb, nichol2017high}.
%% QUANTUM LOGIC
\subsection{Quantum Logic}
In our model, we aim to emulate the Swap and iSwap gate approximatively through Swap-like behaviour by dynamically tuning the system into configuration II as previously described. The iSwap gate is a two-qubit gate that swaps the states of two qubits while introducing a phase factor. It is defined as:
\begin{equation}
    \text{iSwap} = \begin{pmatrix}
    1 & 0 & 0 & 0 \\
    0 & 0 & i & 0 \\
    0 & i & 0 & 0 \\
    0 & 0 & 0 & 1
    \end{pmatrix} \label{eq:iswap_gate},
\end{equation}
where the Swap gate is similarly defined, without introducing the phase factor, as:
\begin{equation}
    \text{Swap} = \begin{pmatrix}
    1 & 0 & 0 & 0 \\
    0 & 0 & 1 & 0 \\
    0 & 1 & 0 & 0 \\
    0 & 0 & 0 & 1
    \end{pmatrix} \label{eq:swap_gate},
\end{equation}
and we have the square root of the Swap gate, $\sqrt{\text{Swap}}$, which maximally entangles the affected qubits, defined as:
\begin{equation}
    \sqrt{\text{Swap}} = \frac{1}{2}\begin{pmatrix}
    1 & 0 & 0 & 0 \\
    0 & 1+i & 1-i & 0 \\
    0 & 1-i & 1+i & 0 \\
    0 & 0 & 0 & 1
    \end{pmatrix} \label{eq:sqrt_swap_gate}.
\end{equation}
The ordinary Swap gate simply exchanges the states of two qubits without introducing any correlations, whereas its square-root variant $\sqrt{\text{Swap}}$, split amplitude coherently between the two qubits, therefore turning a separable state input, $\ket{01}$ or $\ket{10}$, into a maximally entangled Bell state \eqref{eq:bell_states}. These states are indispensable resources for many quantum algorithms and protocols, such as quantum teleportation, entanglement-based cryptography and many error-correcting codes \cite{nielsen2010quantum, bouwmeester1997experimental, yin2020entanglement}. The iSwap gate goes one step further by also inserting a relative phase factor $i$, which not only swaps (or entangles) the states but also provides an extra phase degree of freedom. This phase factor is crucial for implementing certain quantum algorithms and protocols, such as quantum error correction codes, which require precise control over the relative phases of quantum states \cite{tanamoto2008efficient, shor1996fault}. Together, these gates form a versatile toolkit for both coherent state transfer and on-demand entanglement generation in quantum algorithms and communication protocols. 
\\
In our work, configuration $C_{II}$ is designed to bring the first excited states of the two subsystmems into resonance, which allows for a coherent exchange of population between the qubit states. This is due to the Coulomb interaction between the particles at this near-degenerate condition, akin to the avoided crossing shown in Section \ref{sec:avoided_crossings}. Although we will not explicitly model the phase evolution needed to realize the exact Swap or iSwap matrix, the behavior of our system mimics the essential ingredients of an iSwap-like gate: resonant exchange and entanglement generation. \\
By controlling the duration for which the system remains in the various configurations, and the time spent ramping between them, we can identify time windows where the system approximates the desired state-swapping behavior. This is analogous to how iSwap or $\sqrt{\text{iSwap}}$ gates are implemented in superconducting or trapped-ion systems through Hamiltonian engineering and pulse timing \cite{picard2025entanglement}. \\

Thus, our model represents a novel first step towards demonstrating quantum logic operations using Morse-type double well potentials. The anharmonicity of the Morse potential enables isolation of low-lying states and avoids (unwanted) higher-level leakage, which is essential for the construction of robust, tunable qubit architectures\cite{nielsen2010quantum}.
\end{document}