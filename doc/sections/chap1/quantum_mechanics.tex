\documentclass{subfiles}
\begin{document}
\section{Quantum Mechanics}
\subsection*{The quantum mechanical wavefunction and the Schrödinger Equation}
The physical description of any quantum system, i.e the \emph{state space}, is given by the quantum mechanical \emph{wavefunction} (also often called a \emph{state vector})\ref{nielsen2010quantum}, which in Dirac notation\footnote{Named after the physicist Paul Dirac, who was one of the founding fathers of quantum mechanics.} is written as $\ket{\Psi(t)}$. 
This function is a complex-valued function that gives a complete description of both static and dynamic properties of a given quantum system, and thus presents the analogue to the classical notion of a set of trajectories in phase space \ref{hochstuhl2014time}.

The dynamics of the wavefunction is governed by the \emph{Time-dependent Schrödinger Equation} (TDSE),
\begin{equation}
    i\frac{\partial}{\partial t}\ket{\Psi(t)} = H\ket{\Psi(t)}\label{eq:_tdse}
\end{equation}
where $H$ a linear hermitian operator often referred to as the \emph{Hamiltonian}. This operator describes the total energy of the system, and is given by
This equation gives the equation of motion for the wavefunction, and describes how the wavefunction evolves in time.


\subsection*{First quantization}
\subsection*{Second Quantization}
\subsection*{Hartree-Fock}
\subsection*{Morse potential}

\end{document}