\documentclass{subfiles}
\begin{document}
\section{Quantum Mechanics}
\subsection*{The quantum mechanical wavefunction and the Schrödinger Equation}
The physical description of any quantum system, i.e the \emph{state space}, is given by the quantum mechanical \emph{wavefunction} (also often called a \emph{state vector})\ref{nielsen2010quantum}, which in Dirac notation\footnote{Named after the physicist Paul Dirac, who was one of the founding fathers of quantum mechanics.} is written as $\ket{\Psi(t)}$. 
This function is a complex-valued function that gives a complete description of both static and dynamic properties of a given quantum system, and thus presents the analogue to the classical notion of a set of trajectories in phase space \ref{hochstuhl2014time}. 

The dynamics of the wavefunction is governed by the \emph{Time-dependent Schrödinger Equation} (TDSE),
\begin{equation}
    i\frac{\partial}{\partial t}\ket{\Psi(x, t)} = H(x,t)\ket{\Psi(x, t)}\label{eq:_tdse}
\end{equation}
where $H$ a linear hermitian operator often referred to as the \emph{Hamiltonian}. Here $x$ is the position of the particle, and $t$ is the time. 
This operator describes the total energy of the system, and is given by (in atomic units)
\begin{align*}
    H = -\nabla^2 + V(\mathbf{r})
\end{align*}
where $\nabla^2$ is the Laplacian operator, and $V(\mathbf{r})$ is the potential energy of the system, both external and internal.


This equation gives the equation of motion for the wavefunction, and describes how the wavefunction evolves in time.
\subsection*{Second Quantization}
As the earliest formulations of quantum mechanics introduced the ground-breaking concepts of quantized properties like energy, momentum and angular momentum, we can think of this formulation as a "first quantization". As quantum mechanics matured, it became clear that first quantization was not sufficient to describe many-body systems - especially systems where particles are created or annihilated. Second quantization solves this problem by introducing a new mathematical framework, which accounts for both the particle indistinguishability and the creation and annihilation of particles. Furthermore, second quantization allows for a more intuitive and compact notation for many-body systems, where instead of asking "where is particle $i$?", we ask "how many particles are in state $i$?". As such, second quantization is often referred to as the "occupation number representation" of quantum mechanics. This moves the focus from the individual particles to the orbitals they occupy. This reformulation also reduces much of the manipulation of the wavefunction to algebraic operations, which makes numerical implementations much more efficient, and easy to understand.\ref{helgaker2013molecular}.
\\ \\ In second quantization, we introduce a set of creation and annihilation operators, $a^\dagger$ and $a$, which create and annihilate particles in a given state. These operators satisfy the following commutation relations:
\begin{align*}
    [a_i, a_j] = [a^\dagger_i, a^\dagger_j] = 0, \quad [a_i, a^\dagger_j] = \delta_{ij}
\end{align*} 
where $[A, B] = AB - BA$ 
\subsection*{Hartree-Fock}
\subsection*{Morse potential}

\end{document}