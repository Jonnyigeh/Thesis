\documentclass{subfiles}
\begin{document}
\section{Quantum Mechanics}
\subsection*{The quantum mechanical wavefunction and the Schrödinger Equation}
The physical description of any quantum system, i.e the \emph{state space}, is given by the quantum mechanical \emph{wavefunction} (also often called a \emph{state vector})\cite{nielsen2010quantum}, which in Dirac notation\footnote{Named after the physicist Paul Dirac, who was one of the founding fathers of quantum mechanics.} is written as $\ket{\Psi(t)}$. 
This function is a complex-valued function that gives a complete description of both static and dynamic properties of a given quantum system, and thus presents the analogue to the classical notion of a set of trajectories in phase space \cite{hochstuhl2014time}. 

The dynamics of the wavefunction is governed by the \emph{Time-dependent Schrödinger Equation} (TDSE),
\begin{equation}
    i\frac{\partial}{\partial t}\ket{\Psi(x, t)} = H(x,t)\ket{\Psi(x, t)}\label{eq:_tdse}
\end{equation}
where $H$ a linear hermitian operator often referred to as the \emph{Hamiltonian}. Here $x$ is the position of the particle, and $t$ is the time. 
This operator describes the total energy of the system, and is given by (in atomic units)
\begin{align*}
    H = -\nabla^2 + V(\mathbf{r})
\end{align*}
where $\nabla^2$ is the Laplacian operator, and $V(\mathbf{r})$ is the potential energy of the system, both external and internal.


This equation gives the equation of motion for the wavefunction, and describes how the wavefunction evolves in time.
\subsection*{Second Quantization}
As the earliest formulations of quantum mechanics introduced the ground-breaking concepts of quantized properties like energy, momentum and angular momentum, we can think of this formulation as a "first quantization". Here the observables are represented as operators with real eigenvalues and wavefunctions are assigned to individual particles.
As quantum mechanics matured, it became clear that first quantization was not sufficient to describe many-body systems - especially systems with indistinguishable particles, such as fermionic structures, or systems where particles are created or annihilated, such as in chemical reactions, or in the study of elementary particles.\\  
Second quantization solves this problem by introducing a new mathematical framework, which accounts for both the particle indistinguishability and the creation and annihilation of particles, and makes the statefunction also expressed in terms of operators. Furthermore, second quantization allows for a more intuitive and compact notation for many-body systems, where instead of asking "where is particle $i$?", we ask "how many particles are in \emph{state} $i$?". 
As such, second quantization is often referred to as the "occupation number representation" of quantum mechanics. This moves the focus from the individual particles to the orbitals they occupy. This reformulation also reduces much of the manipulation of the wavefunction to algebraic operations, which makes numerical implementations much more efficient, and easy to understand.\cite{helgaker2013molecular}.
\\ \\ In second quantization, we introduce a set of creation and annihilation operators, $a^\dagger$ and $a$, which create and annihilate particles in a given state. An important thing to note, is that these operators differ depending on the type of particles. For bosons, these operators satisfy the following commutation relations:
\begin{align}
    [a_i, a_j] = [a^\dagger_i, a^\dagger_j] = 0, \quad [a_i, a^\dagger_j] = \delta_{ij}\label{eq:commutation}
\end{align}
where $[A, B] = AB - BA$, and for fermions, the \emph{anti-}commutation relations are:
\begin{align}
    \{a_i, a_j\} = \{a^\dagger_i, a^\dagger_j\} = 0, \quad \{a_i, a^\dagger_j\} = \delta_{ij}\label{eq:anti_commutation}
\end{align}
where $\{A, B\} = AB + BA$.
\subsection{Fock Space}
In the framework of second quantization, the concept of a \emph{Fock space}\footnote{First introduced by V. A. Fock in \cite{fock1932konfigurationsraum}} emerge naturally as a mathematical structure for describing quantum systems with variable, or uknown, number of particles. A Fock space is a direct sum of Hilbert spaces, 
\begin{align*}
    \mathcal{F} = \bigotimes_{n=0}^\infty S_{\pm} \mathcal{H}_n
\end{align*}
where each space, $\mathcal{H}_n$ represents a state with fixed a number of particles, and $S_{\pm}$ is the symmetrization operator for bosons ($+$) and fermions ($-$). Meaning, the zero-particle states, one-particle states, two-particle states etc. This encapsulates all possible configurations of a many-body system elegantly. Using the occupation number representation introduced in second quantization, a state in Fock space is not expressed by momenta or position, but rather by the number of particles occupying certain quantum states. \\
For instance, the state $\ket{n_1, n_2, ...}$ informs that $n_1$ particles occupy state $1$, $n_2$ particles  in state $2$. The annihilation and creation operators act on the Fock states by increasing, or decreasing, the occupation numbers of the corresponding states. E.g. the action of the creation operator on a state is given by
\begin{align*}
    a^\dagger_i\ket{n_1, n_2, ...} = \sqrt{n_i + 1}\ket{n_1, n_2, ..., n_i + 1, ...}.
\end{align*}
From this, these operators can describe particle interactions, transitions and dynamics in a many-body system. As the Fock space is constructed by direct sums, two states of different particle numbers are inherently orthogonal 
\\\\ 
For systems of indistinguishable particles, Fock spaces naturally incorporate the Pauli exclusion principle, as the anti-commutation relations \ref{eq:anti_commutation} ensure that no two fermions can occupy the same quantum state. This fundamental property of fermions explains, for example, why electrons in an atom cannot share identical quantum numbers. For bosonic systems (distinguishable particles), the commutation relations \ref{eq:commutation} instead allow multiple particles to occupy the same state, which is crucial for phenomena such as Bose-Einstein condensation.
\subsection*{Hartree-Fock}
Accurately solving the Schödinger equation for many-body systems is a formidable challenge, even in seemingly simple cases such as a one-dimensional system with few interacting, indistinguishable particles. As we mentioned in the section on Hilbert spaces\ref{sec:Hilbert_space}, the dimension of the Hilbert space grows exponentially with the number of particles, making exact solutions computationally infeasible. The inherent complexity arise from the interactions between particles, the Pauli exclusion principle, and the indistinguishability of particles. 




\subsection*{Morse potential}

\end{document}