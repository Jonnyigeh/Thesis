\documentclass{subfiles}
\begin{document}
%%%% HARTREEE FOCKKKKKK
\subsection{Hartree-Fock}\label{app:HF_theory}\textcolor{red}{should move alot of this into an appendix}
Accurately solving the Schödinger equation for many-body systems is a formidable challenge, even in seemingly simple cases such as a one-dimensional system with few interacting, indistinguishable particles. The inherent complexity arise from the interactions between particles, the Pauli exclusion principle, and the indistinguishability of particles. The dimension of the Hilbert space grows exponentially with the number of particles, making exact solutions computationally infeasible. In many cases, such a molecular dynamics and solid-state physics, the Hilbert space is reduced dramatically by imposing the Born-Oppenheimer approximation, which separates the electronic and nuclear motion, effectively disregarding the degrees of freedom of the nuclei by treating the nuclei as fixed. Even so, the many-body problem remains an intractable problem for classical computers, and finding approximate solutions to the Schrödinger equation is therefore necessary. The \emph{Hartree-Fock} method is a fundamental approach for solving the many-body problem in quantum chemistry. \\ \\

%% Many-body concepts
In any many-electron system, the indistinguishability of particles introduce a fundamental contraint on the wavefunction, namely that is must be anti-symmetric under the exchange of any two particles. i.e 
\begin{align*}
    \Psi(\mathbf{r}_1, \mathbf{r}_2, ..., \mathbf{r}_N) = -\Psi(\mathbf{r}_2, \mathbf{r}_1, ..., \mathbf{r}_N)
\end{align*}
This constraint is known as the \emph{Pauli exclusion principle}, which require that no two electrons can occupy the same quantum state. A common way to incorporate this mathematically is to construct wavefunctions using Slater determinants of single-particle orbitals (functions). These orbitals are our basis set of choice $\{\phi_i\}$, and a slater determinant is constructed as follows:
\begin{align*}
    \Psi(\mathbf{r}_1, \mathbf{r}_2, ..., \mathbf{r}_N) = \frac{1}{\sqrt{N!}}\begin{vmatrix}
        \phi_1(\mathbf{r}_1) & \phi_2(\mathbf{r}_1) & \cdots & \phi_N(\mathbf{r}_1)\\
        \phi_1(\mathbf{r}_2) & \phi_2(\mathbf{r}_2) & \cdots & \phi_N(\mathbf{r}_2)\\
        \vdots & \vdots & \ddots & \vdots\\
        \phi_1(\mathbf{r}_N) & \phi_2(\mathbf{r}_N) & \cdots & \phi_N(\mathbf{r}_N)
    \end{vmatrix}
\end{align*}
The mathematical nature of the determinant incorporates the anti-symmetry under particle exchange, as by swapping two columns in a determinant, the sign changes. The Slater determinant are a linear combination of \emph{Hartree products} built from the single-particle orbitals, which are products of spatial orbitals with (or without) the spin orbitals where the spin part is often omitted for simplicity. These single-particle orbitals are the solution of the one-electron Schrödinger equation, 
\begin{align*}
    \hat{h}\phi_i = \epsilon_i\phi_i
\end{align*}
where the full Hamiltonian (for a non-interacting) system would be 
\begin{align*}
    H = \sum_{i=1}^N \hat{h}_i
\end{align*}
which has the solution eigenvector
\begin{align}
    \Psi = \phi_1(\mathbf{r}_1)\phi_2(\mathbf{r}_2)...\phi_N(\mathbf{r}_N)\label{eq:hartree_product}
\end{align}
with corresponding eigenvalue $E = \epsilon_1 + \epsilon_2 + ... + \epsilon_N$, i.e. the sum of single-particle energies. Eq. \ref{eq:hartree_procut} is the Hartree product, and it is the simplest possible wavefunction for a many-body system of non-interacting particles. As is evident, this Hartree product is not anti-symmtric, nor indistinguishable, as the particles are designated a specific orbital to occupy and thus they are distinguishable, which is why the Slater determinant builds linear combinations of such products. In our study, we will make use of both - as our system can be constructed to both exhibit distinguishable and indistinguishable behaviour.\\ \\
Another important concept is the \emph{variational principle}, which states that, for any quantum system, the expectation value of the energy is always greater than, or equal to, the true ground state energy. 
\begin{align}
    E[\Psi] = \frac{\braket{\Psi|H|\Psi}}{\braket{\Psi|\Psi}} \geq E_0 \label{eq:variational_principle}
\end{align}
where $\Psi$ is the trial wavefunction, $H$ is the Hamiltonian operator, and $E_0$ is the true ground state energy of our system. The trial wavefunction in question could be a slater determinant, built from an initial guess for a "good" basis. As previously explained, we may transform this basis using unitary matrices to find a "better" basis, which in this case, would make our energy estimate \emph{lower}. This is the essence of the Hartree-Fock method, where we iteratively improve our basis set to minimize the energy of the system by use of the variational method. For more material and details on the variational method, we refer the reader to chapter 1.3 in \cite{szabo1996modern}.
\\\\
To arrive at the Hartree-Fock equations, we start at the variational principle
\begin{align*}
    E_0 \leq E^{HF} = \bra{\psi^{HF}}H\ket{\psi^{HF}}
\end{align*}
where $\ket{\psi^{HF}}$ is the Hartree-Fock wavefunction, which is a single Slater determinant, and it is normalized so we can omit the denominator in the expectation value. This basis is related to a chosen initial basis by a unitary transformation.
\begin{align*}
    \psi^{HF}_p = \sum_qC_{qp}\psi_q
\end{align*}
where the unitary matrix is expressed by its matrix elements. The Hamiltonian in question is the sum of the kinetic energy operator and the electron-electron repulsion operator, and the energy is given, expressed in the initial basis with the coefficients $C_p$, as
\begin{align*}
    E^{HF} = \sum_i\sum_{pq}C^*_{ip}C_{i}qh_{pq} + \frac{1}{2}\sum_{ij}\sum_{pqrs}C^*_{ip}C^*_{jq}C_{ir}C_{j}su_{pqrs}
\end{align*}
here expressed in terms of the one- and two-electron integrals, in the initial basis. The Hartree-Fock equations are derived by minimizing the energy with respect to the coefficients $C_{ip}$. We take the derivative w.r.t $C_{i}p^*$ which gives us the Hartree-Fock equations
\begin{align*}
    \sum_qh_{pq}C_{iq} + \sum_j\sum_{qrs}C^*_{jr}C_{js}u_{pqrs}C_{iq} = \epsilon^{HF}_{ip}C_{ip}.
\end{align*}
By assumption, the summation over $r$ and $s$ is over all the occupied orbitals (below the Fermi level), as these are the only ones that contribute in the (mean-field) interaction matrix. As a result, the indices reduce to a summation over the occupied orbitals only. This simplification modifies the two-body integrals into $u_{piqi}$, where $i$ runs over orbitals below the Fermi level, while $p,q$ remains the orbitals being varied. This simplification reflects the mean-field approximation central in Hartree-Fock theory, as only interactions with occupied orbitals are considereed. With this, we can now define the Fock operator as
\begin{align*}
    f_{pq} = h_{pq} + \sum_{i<F}C^*_{i}C_{i}u_{piqi}
\end{align*}
and the Hartree-Fock equations can be written as
\begin{align}
    f_{pq}C_{iq} = \epsilon^{HF}_{ip}C_{ip}\label{eq:hf_equations}
\end{align}
which is now a pseduo-eigenvalue equation that we need to solve iteratively. To solve this equation we employ the \emph{Self-Consistent Field} (SCF) procedure, an iterative method designed to converge to the ground-state energy of the system. This procedure starts with an initial guess for the coefficients $C_{ip}^{(0)}$, which defines the initial guess for the molecular orbitals. With this guess, the Fock operator is constructed, and the Hartree-Fock equations are solved. This yields a new set of coefficients $C_{ip}^{(1)}$, and the process is repeated until the HF-energies $\epsilon^{HF}$ converge within a set threshold, or the change in coefficients $C_{ip}$ becomes negligable. A more thorough presentation of the SCF method applied to the Hartree-Fock equations will be presented in later sections. 

\textcolor{red}{Maybe add some more references. What about showing explicitly the integrals? Also the coulomb and exchange terms, maybe these would be good to show. SHould we talk about the Roothan-Hall equations?}

\begin{itemize}
    \item Born-Oppenheimer approximation, herein lies the separation of electronic and nuclear motion and mean-field approximation comes in naturally.
    \item Slater Determinants
    \item Hartree products
    \item Link in orbitals?
    \item Variational principle should be mentioned
    \item 
\end{itemize}

%% METHOD PART OF IT
% \section{Hartree-Fock}\textcolor{red}{This section is somewhat obsolete, we want HF in the appenix, what to do with the implementation?}
% As mentioned in the theory section\ref{sec:HF_theory}, the Hartree-Fock method is an iterative method to obtain optimal basis functions (single particle orbitals) that would minimize the energy, and by the variational method, converge towards the true ground state energy. This is inherently an approximaton, where we imagine that the electrons occupy the lowest possible single-particle orbitals, and it has been proven many times that this is a rather accurate approximation for many systems. \textcolor{red}{(Cite some sources here)}.
% We will now outline the method without much information, before diving into each step in more detail and provide insights into the computational aspects of the method.
% \begin{itemize}
%     \item Construct an initial guess for the single-particle orbitals, $\{\phi_i\}$, often using atomic orbitals or other basis sets.
%     \item Diagonalize the Fock matrix to obtain a new set of orbitals, $\{\phi_i\}$, and calculate Hartree-Fock energy, $E_{HF}$.
%     \item Repeat the process until the energy converges, i.e. the change in energy between iterations is below a certain threshold.
%     \item Calculate the total energy of the system, and the electron density, and use this to calculate other properties of the system.
% \end{itemize}
% The first step in the Hartree-Fock procedure is to define an initial ansatz for the trial wavefunction, which is typically represented as a Slater determinant, as discussed earlier. This ansatz requires an initial choice of basis set, which plays a crucial role in the overall success of the method. The initial basis set serves as a starting point for constructing orbitals and significantly influences the convergence of the self-consistent field (SCF) procedure.\\  
% \\ Choosing an appropriate basis set is a nuanced and complex task. A well-chosen basis can simplify numerical calculations and accelerate convergence, whereas a poor choice may lead to slow convergence or even failure to converge. 
% \\ \\


% There are a multitude of basis sets to choose from, and the choice is guided by the nature of the system. For instance, in quantum dots systems with strong confinement, the quantum harmonic oscillator basis sets are often used to great success\cite{Yuan_2017}. A different procedure to using pre-defined basis sets, is to solve the Schrödinger equation for the non-interacting system, and use these single-particle orbitals as the initial basis set. The latter may often yield quicker convergence due to the functions being specifically tailored for the potential, but at the cost of more computational resources, and in some cases may not even be possible. 
% \\
% \\ With the initial basis, we construct the Fock matrix, which is a matrix representation of the Fock operator in the basis of the single-particle orbitals as seen in section \ref{sec:HF_theory}. The Fock matrix is given by
% \begin{align}
%     F_{pq} = h_{pq} + \sum_{i,j}^N u_{piqj}\rho_{ij} \label{eq:fock_matrix}
% \end{align}
% where $\rho_{ij}$ is the density matrix formed from the eigenvectors of the anti-symmetrized system hamiltonian (\textcolor{red}{Dette bør nok nevnes i seksjon 1.2.1 Hartree-Fock, og refereres til hvordan interaksjonleddet gjørs anti-symmtrisk $u = u_ijab - u_ijba$}). The following snippet 
% \begin{lstlisting}[language=Python]
% h = ...
% u = ...
% def fill_fock_matrix(C):
%     fock = np.zeros(h.shape, dtype=np.complex128)
%     density_matrix = np.zeros((h.shape[0],h.shape[0]), dtype=np.complex128)
%     for i in range(n_particles):
%         density_matrix += np.outer(C[:, i], np.conj(C[:, i]))
%     fock = np.einsum('ij, aibj->ab', density_matrix, u, dtype=np.complex128)        # Compute the two-body operator potential
%     fock += h                                                                       # Add the one-body operator hamiltonian

%     return fock
% \end{lstlisting}

% \textcolor{red}{not finished}

\end{document}