\documentclass{subfiles}
\begin{document}
\section{Time-Independent Basis for the Morse Double-Well}\label{sec:time_independent_basis}


\subsection{Exchange interaction in the basis state representation}
In our bipartite Hartree approach (Section \ref{sec:bipartite_hartree}) we include only the direct Coulomb interaction term between the particles, disregarding the exchange interaction as we aim to construct our system in such a way that particles are strictly localized, and thus, distinguishable. Even though the exchange interaction does not generate entanglement, it shifts single-particle energies and can alter the dynamics and composition of our reduced basis. Here we wish to quantify the significance (or insignificance) of the exchange term to validate our assumption of distinguishability. We will investigate the ground state energy of the system computed with and without the exchange interaction, and compare the two, as we've outlined in section \ref{sec:distinguishability}. \\ 

Recall that in the Hartree product-state approach ('distinguishable particles'), the two-particle Hamiltonian
\begin{align*}
    H_{dist} = H_L \otimes \mathbb{I} + \mathbb{I} \otimes H_R + V_{Coulomb},
\end{align*}
where $H_L$ and $H_R$ are the single-particle Hamiltonians of the left and right wells, respectively, and $V_{Coulomb}$ is the direct Coulomb interaction between the particles. In contrast, the Hartee-Fock approach ('indistinguishable particles') includes the exchange interaction, which is represented by the antisymmetrized two-particle Hamiltonian
\begin{align*}
    H_{HF} = \sum_{pq} h_{pq} a_p^\dagger a_q + \frac{1}{2} \sum_{pqrs} V_{pqrs} a_p^\dagger a_q^\dagger a_r a_s,
\end{align*}
where $h_{pq}$ are the single-particle matrix elements, and $V_{pqrs} = J_{pqrs} - K_{pqrs}$ are the Coulomb interaction matrix elements, with $J_{pqrs}$ being the direct and $K_{pqrs}$ the exchange interaction. The overview of the Hartee-Fock procedure can be found in section \ref{sec:hartree_fock} and more detailed in appendix \ref{app:hartree_fock}. \\\\
We diagonalize both Hamiltonians in their respective bases and compare the ground state energies. In the distinguishable-particle case, we note that we project only onto the anti-symmetric energy states for a more direct and accurate comparison with the indistinguishable-particle case. This is done for the following well separations on a grid of length $400$, with $4001$ and $800$ grid points for energy eigenbasis (Hartree-Fock) and Sinc-DVR base (Hartree) respectively, and we calculat the energy shift
\begin{align*}
    \Delta E = E_{HF} - E_{dist},
\end{align*}
where $E_{HF}$ is the ground state energy of the Hartree-Fock Hamiltonian and $E_{dist}$ is the ground state energy of the Hartree Hamiltonian. The results are shown in figure \ref{fig:exchange_shift}. 
\begin{figure}[h!]
    \centering
    \includegraphics[width=1.0\textwidth]{figs/exchange_shift.pdf}
    \caption{Ground-state energy as a function of the inter-well separation $d$ for the Morse double-well potential. The rapid decay of the exchange term of the coulomb interaction matrix with increasing well separation indicates that this term has a negligible effect on the ground state energy, confirming our assumption of locality and the usage of product states to express the wavefunction, as the particles become distinguishable for appropriate separation. We can see that the exchange term is close to zero $eV$ for well separations larger than $d = 10$a.u. For similar separations, our product state Hartree procedure converge with the truncated CI solution. }
    \label{fig:exchange_shift}
\end{figure}
We observe that the exchange interaction has an almost zero contribution for moderate separations. This confirms our assumption that the particles are distinguishable in our system, and we can safely neglect the exchange interaction in our simulations. The potential parameters used for this simulations are the optimal parameters for the Morse double-well configuration $C_I$ \eqref{eq:C_I_parameters}, as described in section \ref{sec:optimization_procedure} and presented in Section \ref{sec:optimization_result}. \textcolor{red}{(TODO: Add more discussion on the results, and how this relates to the distinguishability of the particles, and the validity of our approach. Note that Hartree energy is non-variational due to lack of anti-symmetry DISCUSS THIS)} \\


\end{document}