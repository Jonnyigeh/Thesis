\documentclass{subfiles}
\begin{document}
% \section{Hartree-Fock}\textcolor{red}{This section is somewhat obsolete, we want HF in the appenix, what to do with the implementation?}
% As mentioned in the theory section\ref{sec:HF_theory}, the Hartree-Fock method is an iterative method to obtain optimal basis functions (single particle orbitals) that would minimize the energy, and by the variational method, converge towards the true ground state energy. This is inherently an approximaton, where we imagine that the electrons occupy the lowest possible single-particle orbitals, and it has been proven many times that this is a rather accurate approximation for many systems. \textcolor{red}{(Cite some sources here)}.
% We will now outline the method without much information, before diving into each step in more detail and provide insights into the computational aspects of the method.
% \begin{itemize}
%     \item Construct an initial guess for the single-particle orbitals, $\{\phi_i\}$, often using atomic orbitals or other basis sets.
%     \item Diagonalize the Fock matrix to obtain a new set of orbitals, $\{\phi_i\}$, and calculate Hartree-Fock energy, $E_{HF}$.
%     \item Repeat the process until the energy converges, i.e. the change in energy between iterations is below a certain threshold.
%     \item Calculate the total energy of the system, and the electron density, and use this to calculate other properties of the system.
% \end{itemize}
% The first step in the Hartree-Fock procedure is to define an initial ansatz for the trial wavefunction, which is typically represented as a Slater determinant, as discussed earlier. This ansatz requires an initial choice of basis set, which plays a crucial role in the overall success of the method. The initial basis set serves as a starting point for constructing orbitals and significantly influences the convergence of the self-consistent field (SCF) procedure.\\  
% \\ Choosing an appropriate basis set is a nuanced and complex task. A well-chosen basis can simplify numerical calculations and accelerate convergence, whereas a poor choice may lead to slow convergence or even failure to converge. 
% \\ \\


% There are a multitude of basis sets to choose from, and the choice is guided by the nature of the system. For instance, in quantum dots systems with strong confinement, the quantum harmonic oscillator basis sets are often used to great success\cite{Yuan_2017}. A different procedure to using pre-defined basis sets, is to solve the Schrödinger equation for the non-interacting system, and use these single-particle orbitals as the initial basis set. The latter may often yield quicker convergence due to the functions being specifically tailored for the potential, but at the cost of more computational resources, and in some cases may not even be possible. 
% \\
% \\ With the initial basis, we construct the Fock matrix, which is a matrix representation of the Fock operator in the basis of the single-particle orbitals as seen in section \ref{sec:HF_theory}. The Fock matrix is given by
% \begin{align}
%     F_{pq} = h_{pq} + \sum_{i,j}^N u_{piqj}\rho_{ij} \label{eq:fock_matrix}
% \end{align}
% where $\rho_{ij}$ is the density matrix formed from the eigenvectors of the anti-symmetrized system hamiltonian (\textcolor{red}{Dette bør nok nevnes i seksjon 1.2.1 Hartree-Fock, og refereres til hvordan interaksjonleddet gjørs anti-symmtrisk $u = u_ijab - u_ijba$}). The following snippet 
% \begin{lstlisting}[language=Python]
% h = ...
% u = ...
% def fill_fock_matrix(C):
%     fock = np.zeros(h.shape, dtype=np.complex128)
%     density_matrix = np.zeros((h.shape[0],h.shape[0]), dtype=np.complex128)
%     for i in range(n_particles):
%         density_matrix += np.outer(C[:, i], np.conj(C[:, i]))
%     fock = np.einsum('ij, aibj->ab', density_matrix, u, dtype=np.complex128)        # Compute the two-body operator potential
%     fock += h                                                                       # Add the one-body operator hamiltonian

%     return fock
% \end{lstlisting}

% \textcolor{red}{not finished}


%%% HARTREE METHOD
\section{Hartree method}\label{sec:Hartree_method}
As presented in section \ref{sec:hartree_theory}, the Hartree method is a self-consistent field method (SCF), where we solve the coupled eigenvalue equations for the two subsystems (particles) iteratively. The method is very similar to the Hartree-Fock method, and the main difference is that we do not include an exchange term and the wavefunction itself a single hartree product state. The general method is as follows:
\begin{algorithm}[ht]
\caption{Self-Consistent Hartree Procedure}
\medskip
\noindent\textbf{1. Initial guess:}\\
\quad Construct a starting set of single-particle orbitals \(\{\chi_i\}\).  

\medskip
\noindent\textbf{2. Iterate to self-consistency:}\\
\quad a. Build the Hartree matrix \(f\) from the one- and two-body integrals weighted by the current \(\{\chi_i\}\).\\
\quad b. Diagonalize \(f\) to obtain updated orbitals \(\{\phi_i\}\) and orbital energies \(\{\epsilon_i\}\).\\
\quad c. Compute the total Hartree energy \(E\).\\
\quad d. If \(\bigl|E - E_{\rm previous}\bigr| < \delta E\), stop; otherwise set \(\chi_i \leftarrow \phi_i\) and repeat.

\medskip
\noindent\textbf{3. Final evaluation:}\\
\quad Using the converged orbitals, calculate the final total energy, and any other desired observables.\label{alg:Hartree}
\end{algorithm}

We make an initial ansatz for the single-particle orbitals by constructing the Hartree matrix as in \eqref{eq:hartree_matrix}, and diagonalize the matrix to obtain the new set of orbitals, one set for each subsystem. The initial Hartree matrix is thus constructed as
\begin{align*}
    f_{\alpha\beta}^{M(0)} = h_{\alpha\beta}^M
\end{align*}
where $M$ is the subsystem index, and $\alpha, \beta$ are the basis functions. This yields our initial set of orbitals, but does not include any interaction between the two subsystems. We then, in the next iterations construct new Hartree matrices, and diagonalize them to obtain new orbitals. The new Hartree matrix for the i'th iteration is constructed as
\begin{align*}
    f_{\alpha\beta}^{M(i)} = h_{\alpha\beta}^M + \sum_{i}^{N} u_{\alpha\gamma\beta\delta}^{M} \rho_{\gamma\delta}^{M},
\end{align*}
where the density matrix $\rho_{\gamma\delta}^{M}$ is constructed from the previous set of orbitals, and used to calculate the interaction term \emph{for subsystem $M$ specifically}. This differs from Hartree-Fock where the interaction term is calculated from the total density matrix ($\rho^A\otimes\rho^B$). In the following code snippet the density matrix is not used directly, but an equivalent einstein summation is performed, following the equations \eqref{eq:bipartite_hartree} in section \ref{sec:bipartite_H}.
\begin{lstlisting}[language=Python]
def construct_hartree_matrices(self, h_l, h_r, u_lr, c_l, c_r):
    return (
        h_l + np.einsum('j, ijkl, l -> ik', c_r[:,0].conj(), u_lr, c_r[:,0]),
        h_r + np.einsum('i, ijkl, k -> jl', c_l[:,0].conj(), u_lr, c_l[:,0]),
    )
def diagonalize_hartree_matrices(self, f_l, f_r):
    eps_l, c_l = scipy.linalg.eigh(f_l, subset_by_index=[0, self.num_basis_l - 1])
    eps_r, c_r = scipy.linalg.eigh(f_r, subset_by_index=[0, self.num_basis_r - 1])
    return eps_l, c_l, eps_r, c_r
\end{lstlisting}
With these two functions we set up the SCF as outlined in Algorithm \ref{alg:Hartree}. We use the \texttt{scipy.linalg}\cite{2020SciPy-NMeth} package and its hermitian eigensolver \texttt{eigh} to diagonalize the Hartree matrices, selecting the subset of functions we wish to extract, corresponding to the number of available basis functions in each subsystem. In our implementation we terminate the SCF procedure when the single-particle Hartree energies converges within a threshold of $\Delta E < 10^{-10}$ a.u., or we reach a maximum numbr of iterations, which we set to 1000. For most Morse double-well configurations, the SCF procedure converges within a few 10-20 iterations starting from the non-interacting eigenstates. We rarely experience divergence, but if it should occur, we would recommend starting from a different initial guess to accelerate convergence. 


\section{Configuration Interaction}\label{sec:CI_method}
Building on the Hartree method, our CI implementation projects the full two-particle product state Hamiltonian into a smaller subspace of anti-symmetrized basis states. Recall from Section \ref{sec:CI}, that in practice we do the following:
\begin{itemize}
    \item Truncate to the lowest $N$ ($N<M$) orbitals in each well.
    \item Form all unordered pairs of these orbitals, i.e. $\{\phi_i, \phi_j\}$ where $i \neq j$.
    \item Build the CI Hamiltonian by calculating the matrix elements of the two-particle Hamiltonian in this reduced basis.
    \item Diagonalize the CI Hamiltonian to obtain the energy eigenstates and eigenvalues.
\end{itemize}
In practice, to perform the CI procedure, we build a linear map $P$ that projects any vector in the $M$-dimensional product-state Hilbert space down onto our anti-symmtric subspace of dimension $\binom{N}{2}$. Concretely, this means that we enumrate all possible unordered pair of orbitals with $0\leq i < j < N$, and for each pair construct the anti-symmetric basis state
\begin{align*}
    \ket{\phi_i, \phi_j} = \frac{1}{\sqrt{2}}(\ket{\phi_i}\otimes\ket{\phi_j} - \ket{\phi_j}\otimes\ket{\phi_i}).
\end{align*}
This flattened representation of the anti-symmetric basis state will build the columns of the projection matrix $P$, which ensures $P^\dagger P = \mathcal{I}$, where $\mathcal{I}$ is the identity matrix. 

Once $P$ is constructed, we can project the full two-particle Hamiltonian $H$ onto the anti-symmetric subspace by
\begin{align*}
    H_{\text{CI}} = P^\dagger H_{\text{prod}} P.
\end{align*}
Because $H_{\text{prod}}$ contains all the one- and two-body terms of the product basis, the projected Hamiltonian $H_{\text{CI}}$ will contain all the relevant interactions between the two particles in the anti-symmetric subspace. This gives us a reduced Hamiltonian that can be diagonalized to obtain the energy eigenstates and eigenvalues of the system. The CI method thus allows us to efficiently explore the low-energy spectrum of the two-particle system while maintaining the anti-symmetry required by quantum mechanics. 

The following code snippet illustrates how we construct the projection matrix $P$ and the CI Hamiltonian $H_{\text{CI}}$ in practice:
\begin{lstlisting}[language=Python]
h_l = ... # One-body Hamiltonian left well
h_r = ... # One-body Hamiltonian right well
u_lr = ... # Two-body coulomb matrix elements
H = np.kron(h_l, np.eye(M)) + np.kron(np.eye(M), h_r) + u_lr 
M = h_l.shape[0]               # number of one-particle functions
pairs = [(i, j) for i in range(M) for j in range(i)]

P = np.zeros((M*M, len(pairs)), dtype=complex)
for idx, (i, j) in enumerate(pairs):
    e_i = np.zeros(M); e_i[i] = 1
    e_j = np.zeros(M); e_j[j] = 1
    psi_ij = np.kron(e_i, e_j)
    psi_ji = np.kron(e_j, e_i)
    P[:, idx] = (psi_ij - psi_ji) / np.sqrt(2)
H_CI = P.conj().T @ H @ P  # Projected CI Hamiltonian
eps_CI, C_CI = np.linalg.eigh(H_CI)  # Diagonalize CI Hamiltonian
\end{lstlisting}
This code constructs the projection matrix $P$ by iterating over all pairs of orbitals, creating the anti-symmetric basis states, and then projecting the full two-particle Hamiltonian onto this subspace. The resulting CI Hamiltonian $H_{\text{CI}}$ can then be diagonalized to obtain the energy eigenstates and eigenvalues of the system in the anti-symmetric subspace.

\end{document}